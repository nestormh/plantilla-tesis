%%
%% agradecimientos.tex - Memoria de la tesis
%%
%%   Copyright 2009-2010 Jesús Torres <jmtorres@ull.es>
%%
%% Esta obra está bajo licencia Creative Commons Reconocimiento 3.0 Unported
%%
\cleardoublepage
\thispagestyle{empty}
\addsec*{\protect\centering Agradecimientos}
\markboth{Agradecimientos}{}

En primer lugar, me gustaría agradecer a mis directores de Tesis, Dr. D. Leopoldo Acosta Sánchez y Dr. D. Jonay Tomás Toledo Carrillo, la oportunidad de colaborar en este proyecto, sin el cual no podría haber aprendido lo que he aprendido, ver lo que visto, ni hacer lo que entre todos hemos hecho. Y por supuesto, los buenos consejos y el tiempo prestado, que no ha sido poco, especialmente en las últimas semanas de la Tesis.

No puedo quedarme sin dar mi agradecimiento al Dr. D. Jesús Javier Espelosín Ortega, D. Rafael Arnay del Arco y D. Daniel Perea Ström (¡equipo MARANEDA!), y D. Jonatán Felipe, con los que empecé este camino de cabras que parece no terminar nunca. Nadie nos contó que estaba sin asfaltar, pero estuvimos lo suficientemente locos como para seguir. Agradezco el apoyo ofrecido todos estos años, y sobre todo el buen humor, que nunca ha faltado. Igualmente, agradezco al Dr. D. Jesús Miguel Torres Jorge el haber malgastado parte de su valioso tiempo resolviendo infinitas dudas acerca de todo lo que no era capaz de echar a andar por mí mismo. 

Cómo no, al ``equipo cafeína'', por haber llevado la procastinación a niveles desconocidos. Ellos son: Dña Ángela Hernández López, D. Antonio Morell González, D. Ayoze Marrero Ramos, Dña. Elena Santos Hernández, D. Esteban Rodríguez, D. Eusebio Morell González, D. Francisco Fumero Batista, Dr. D. Iván Castilla Rodríguez, D. Javier Hernández Aceituno, Dr. D. Juan Albino Méndez Pérez (a.k.a. Alexis), Dña. Mariana Cairós González, D. Pedro Antonio Toledo Delgado, Dña. Sara González Pérez, Dña. Silvia de León, Dña. Silvia Vera González, D. Yeray Callero de León, y Dña. Kelin Victoria Zúñiga Meneses. Igualmente, al resto del ya extinto Departamento de Ingeniería de Sistemas y Automática y Arquitectura y Tecnología de Computadores (ISAATC): Dra. Dña. Silvia Alayón Miranda, Dra. Dña. Rosa María Aguilar Chinea, D. Ginés Coll Barbuzano, Dr. D. José Ignacio Estévez Damas, Dra. Dña. Carina Soledad González González, Dr. D. Evelio José González González, Dr. D. Alberto Francisco Hamilton Castro, Dr. D. Sergio Elías Hernández Alonso, Dr. D. Graciliano Nicolás Marichal Plasencia, Dr. D. Roberto Luis Marichal Plasencia, D. Juan Julián Merino Rubio, Dr. D. Lorenzo Moreno Ruiz, Dra. Dña. Vanesa Muñoz Cruz, Dr. D. Sid Ahmed Ould Sidha, Dr. D. José Demetrio Piñeiro Vera, D. Héctor Javier Reboso Morales, Dr. D. José Luis Sánchez de la Rosa, Dr. D. José Francisco Sigut Saavedra y Dra. Dña. Marta Sigut Saavedra.

Vorrei ringraziare il Proffesore Sig. Alberto Broggi per l'opportunità di collaborare e imparare in VisLab, e anche Dott. Paolo Zani e Dott. Mirko Felisa per aver accettato di rivedere questa noiosa tesi e il suo incondizionato supporto durante il mio soggiorno a Parma. Inoltre, apprezzo il consiglio e la grande ospitalità dimostrata da Dott. Luca Bombini, Sig. Michele Buzzoni, Sig. Gabriele Camellini, Dott. Pietro Cerri, Sig. Alessandro Coati, Dott. Rean Isabella Fedriga, Sig. Alessandro Giacomazzo, Dott. Paolo Grisleri, Dott. Paolo Medici, Dott. Pier Paolo Porta, Sig. Mario Sabbatelli e Sig. Pietro Versari, che mi hanno fatto sentire a casa. Anche Sig. Moisés Díaz Cabrera, per la compagnia.

I must thank Prof. Luc Van Gool for allowing me staying some months in PSI/VISICS, acquiring the expertise of his group. I would also like to extend the thanks for the help and the good mood to Mr. Roeland De Geest, Mr. Bert Deknuydt, Mr. Vincent De Smet, Mr. Basura Fernando, Mrs. Rosalia Galiazzi Schneider, Mr. Stam Georgoulis, Mr. Amir Ghodrati, Dr. Hakan Bilen, Mr. Xu Jia, Mr. Jan Knopp, Mr. Paul Konijn, Dr. Marco Pedersoli, Dr. Markus Mathias, Mr. Andelo Martinovic, Mr. Wim Moreau, Mr. José Antonio Oramas Mogrovejo, Mr. Konstantinos Rematas, Mr. Gilad Sharir, Dr. Radu Timofte, Dr. David Tingdahl and Dr. Tatiana Tommasi.

He dejado para el final a los que siempre han estado ahí. A Emilio Brito López, por haberme cuidado la piedra y haberme ayudado a encontrar el océano; a Beatriz Santos Hernández, por demostrar que 360 no es divisible por 5; y a Miguel Ángel Yonte Rodríguez, por recordarme que vivimos bajo la amenaza constante de que un asterisco nos caiga sobre la cabeza. No me olvido de Lilia Ana Ramos y sus consejos sobre los efectos secundarios de la amapola, Jessica Luis López y su inapropiada afición al surf carnavalero, o Ettore Gendusa, sin cuya colaboración esta sección de agradecimientos podría haber acabado en desastre. De Zebenzui Álvarez Lugo aprendí a encontrarle el punto a las cosas, y junto a Jorge Martín Afonso busqué todo lo negro.

Un reconocimiento especial a mi familia, quienes siempre han estado ahí apoy- ándome, animándome y, sobre todo, educándome. No puedo ignorar el hecho de que si no fuera por ellos, no podría haber llegado a lo que he llegado, ni podría haber disfrutado de ciertos privilegios de los que me siento muy agradecido. De mi madre, Inmaculada Hernández Gil, heredé el gusto por viajar, descubrir nuevos horizontes y llegar siempre un poco más lejos; de mi padre, Bernardo Morales Trujillo, aprendí el gusto por conocer, por aprender y por hacer bien las cosas. Este reconocimiento también va para mi hermana, Esther Morales Hernández, que aunque haya puesto mar de por medio, siempre será mi experta en series y cultura contemporánea favorita.

El último agradecimiento lo he reservado para la Doctora Esther Sanromá Ramos (Uchy para los amigos), la persona más importante en mi vida, y con la que más me alegro de haberme cruzado. Agradezco sus esfuerzos por convertirme en una persona socialmente aceptable, la tolerancia demostrada ante mis contínuos \emph{festivales del humor} y su paciencia a la hora de hacerme ver diversos errores de cálculo espaciales (1.35\,m y 1.50\,m no son lo mismo), temporales (10 minutos y media hora no son medidas equivalentes) y espectrales (aunque los árboles sean así, el marrón y el verde por lo visto no combinan). Pero sobre todo, agradezco los buenos momentos y la felicidad que ha sido capaz de darme.








