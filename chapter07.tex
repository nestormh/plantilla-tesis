%%
%%  chapter07.tex - Obstacle Detection and Planning for Autonomous Vehicles based on Computer Vision Techniques
%%
%%  Copyright 2014 Néstor Morales <nestor@isaatc.ull.es>
%%
%%  This work is licensed under a Creative Commons Attribution 4.0 International License.
%%

\graphicspath{{./images/chapter07/bmps/}{./images/chapter07/vects/}{./images/chapter07/}}

\chapter{Local Planning}\label{ch:chapter07}

Intro:
Path planning allows an autonomous vehicle to determine the behavior of a vehicle by itself.

Ya tenemos un conocimiento exhaustivo del entorno del vehículo y una ruta global que conecta su posición con la del punto de destino
Con esta ruta global por sí sola no hacemos nada, ya que no modela la impredictibilidad del entorno a largo plazo, ya que este entorno es cambiante debido a la presencia de obstáculos móviles.
Además necesitamos un mecanismo que nos permita seguir la trayectoria.

Muchos de estos métodos de generación trayectorias se basan en un esquema de optimización discreta:

Un cjto finito de paths es calculado:
+ reduce el espacio de soluciones y
+ permite una implementación de tiempo real 


Habitualmente esto se hace mediante forward integration de las ecuaciones diferenciales que describen la dinamica del vehiculo
Del cjto de trayectorias se escoge aquella que minimiza un det. coste
Para la generacion de trayectorias un modelo parametrico es empleado, habitualmente func. polinomiales de un orden arbitratio. 
Nosotros usamos splines calculados sobre el espacio de frenet


La ruta global forma un base frame de un sistema de coordenadas curvilineo que es el espacio de generacion de las paths

La info. direccional de la ruta global es incorporada a la de maniobras del vehiculo ajustando el offset lateral al base frame

El trabajo está inspirado en el sistema desarrollado por los coreanos y stanley

- empleamos un base frame
- generamos rutas candidatas basadas en la posición actual respecto al base frame y una serie de endpoints en posiciones fijadas a diferentes offsets respecto al base frame
- Cada ruta es evaluada cuantitativamente mediante una funcion de coste
  + Al igual que en Stanley, nosotros usamos la distancia al centro de la ruta como funcion de coste
    ++ A diferencia suya, el centro de la ruta no es calculado mediante un metodo basado en filtros de Kalman, sino el metodo basado de multiclass SVM descrito en el capitulo anterior
  + Al igual que los coreanos, nuestra funcion de coste se basa en la suavidad de la ruta y el coste de chocar con los obstaculos calculado como una funcion de distancia a los mismos
    ++ Sin embargo, en nuestro caso este coste no se basa en el blurring de los datos binarios de un mapa de obstaculos
      +++ Nosotros usamos un mapa de costes basados en una funcion exponencial relativa a la distancia y al footprint del vehículo
  + Tambien usamos las funciones de coste...

Estas funciones de coste seran descritas con mayor detalle en la seccion XXX

En este trabajo nos hemos focalizado en la consecucion de resultados experimentales que logren el correcto funcionamiento de verdino a nivel practico, mas que en mejorar el algoritmo en si mismo, aunque aportamos algunas soluciones que mejoran los resultados iniciales al ser adaptado a las caracteristicas especiales de nuestro vehiculo.

El algoritmo resultante ha sido aplicado sobre verdino. Al final de este capitulo se muestran algunos resultados experimentales. Ademas, hay disponible una serie de videos que muestran el vehiculo en funcionamiento.
  
===========================================================================
Poner la parte de generacion del mapa de costes
  Describir brevemente, esto mejor lo dejo para el ultimo capitulo
Hablar de la parte de frenet
  Decir que hay codigo disponible de esta parte
Hablar de los costes
  Asociar los originales con el paper de Corea
  Asociar la distancia lateral con Stanley
  Describir los nuevos costes
Hablar de como calculamos la ruta final
  Describir el modelo de la bicicleta
Resultados
  Algunas situaciones complejas en rviz
  Mostrar situaciones reales en imagenes
  Enlazar videos
Resumen
  Hablar de las ventajas/desventajas
  Indicar que en la fase siguiente se describe la integracion con los obstaculos detectados empleados por los metodos de los capitulos previos
  
