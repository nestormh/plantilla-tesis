%%
%% notacion.tex - Memoria de la tesis
%%
%%   Copyright 2009-2010 Jesús Torres <jmtorres@ull.es>
%%
%% Esta obra está bajo licencia Creative Commons Reconocimiento 3.0 Unported
%%
\chapter*{Símbolos y notación}\label{notación}
\pdfbookmark{Símbolos y notación}{notación}
\markboth{Símbolos y notación}{}
\begin{longtable}{rp{0.8\textwidth}}
  $a$               & escalar \\
  $|a|$             & valor absoluto de $a$ \\
  $\vec{a}$         & vector columna \\
  $a_i$             & $i$-ésimo elemento del vector $\vec{a}$ o del conjunto
    de escalares $\{a_n\}$  \\
  $\vec{1}$         & vector columna $[1,1,\ldots,1]^T$ \\
  $\|\vec{a}\|$     & norma euclídea de $\vec{a}$  \\
  $\vec{a}^T$       & traspuesta de $\vec{a}$ \\
  $\mat{A}$         & matriz  \\
  $\vec{a}_i$       & $i$-ésimo elemento de conjunto de vectores $\{\vec{a}_n\}$
    o $i$-ésima columna de la matriz $\mat{A}$ \\
  $a_{ij}$          & elemento en la fila $i$-ésima y columna $j$-ésima de
    la matriz $\mat{A}$ \\
  $\mat{I}$         & matriz identidad \\
  $\mat{A}^{-1}$    & inversa de la matriz $\mat{A}$ \\
  $\|\mat{A}\|_F$   & norma de Frobenius de la matriz $\mat{A}$ \\
  $\diag(\vec{a})$  & matriz diagonal cuyo $i$-ésimo elemento de la diagonal
    es $a_i$ \\
  $\det(\mat{A})$   & determinante de la matriz $\mat{A}$ \\
  $\trace(\mat{A})$ & traza de la matriz $\mat{A}$ \\
  $[\ldots]$        & vector o matriz \\
  $\{\ldots\}$      & conjunto o lista de elementos \\  
  $X, Y, Z$         & ejes de coordenadas en $\spc{R}^3$ \\
  $x, y$            & ejes de coordenadas en $\spc{R}^2$ \\
  $\spc{F}$         & espacio de características \\
  $\func{f}(x)$     & función $\func{f}$ en $x$ \\
  $\func{f}(x;p)$   & función $\func{f}$ en $x$ con parámetro $p$\\
  $\hat{a}$         & estimación de $a$ \\
  $\langle{a}\rangle$ & media del conjunto $\{a_n\}$ \\
  $\tilde{a}_i$     & $i$-ésimo elemento del conjunto $\{a_n\}$ al que se le ha
    restado la media de dicho conjunto \\
  $a^{(t)}$         & valor de $a$ en la iteración $t$ \\
  $\var(a)$         & varianza de $a$ \\
  $\dist{N}(\vec{\mu},\UPSIGMA)$  & distribución normal multivariable de media
    $\vec{\mu}$ y varianza $\UPSIGMA$ \\
  $\func{D}_{KL}(\dist{P}\|\dist{Q})$ & divergencia de Kullback-Liebler entre
    las distribuciones de probabilidad $\dist{P}$ y $\dist{Q}$
\end{longtable}
